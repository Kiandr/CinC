% This file has been slightly modified from Hans Kestler's original to reflect
% changes in the CinC style file (cinc.cls), in which the width of the text
% columns has been slightly reduced.  Accordingly, the size of figure 2 has
% been reduced so that the entire paper will fit into four pages as in the
% original. --GBM
\documentclass[twocolumn]{cinc}


\usepackage{epsfig}



\def\IR{{\rm I\! R}}
\def\IN{{\rm I\! N}}
\def\IF{{\rm I\! F}}
\def\IP{{\rm I\! P}}
\def\IK{{\rm I\! K}}
\def\IZ{{\rm Z\!\! Z}}
\def\IQ{\,{\rm Q\!\!\! \vrule width 0.5pt height 7pt depth 0pt }\;\,}
\def\IC{\,{\rm C\!\!\!\>\! \vrule width 0.5pt height 7pt depth 0pt }\;\,}
\def\argmin{\rm argmin}

\sloppy



\begin{document}

\bibliographystyle{cinc}

\title{Photoplethysmography  as Primary Tool for Progressive Haemorrhage Assesment During Progressive Lower Body Negative Pressure}
\author{Kian~Davoudi$^{1}$, Kouhyar~Tavakolian$^2$, Bozena~Kaminska$^1$ 
\\[1em]
        $^1$Simon Fraser University, Vancouver, Canada \\[.25ex]
        $^2$University of North Dakota, North Dakota, United Stated of Amercia\\[1em]
}

\maketitle






\begin{abstract}
Lack of a precise index to identify degree of  haemorrhage prior cardiovascular collapse is being considered a driving cause of fatality in patients experiencing progressive haemorrhage. The traditional clinical haemorrhage assessment consists of conventional beat-to-beat heart rate and arterial blood pressure measurement. But, as shown in xx, prior cardiovascular collapses,  these indexes continue projecting no significant shift in their status. Hence, Convertino and Tavakolian have shown that SCG and PPG indeed provide such a window to alarm cardiovascular collapse prior radical projection on conventional measurements. Hence, As I shown in xxx, a mechanically flexible PPG sensor was designed and optimized for continuous monitoring of cardiovascular and respiratory performance. But, the reliability of the this sensor for calcification of stages of haemorrhage, was not assessed. Thus, I aimed to resolve this issue,by conducting Lower Body Negative Pressure Test to stimulate haemorrhage in 10 awaked patients and compared the reliability of PPG signal captured from mechanically flexible sensor, where due to arterial blood loss in upper body was diminished, with conventional PPG captured from FDA approved pulse oximeter. Results from Bland and Altman analysis of  a minutes of PPG signal at -50 mm for 10 subjects were, 


\end{abstract}


\section{Introduction}
\subsection{PhotoPlethysmography}
Many pulse oximetry technologies display a processed and filtered representation of the photoplethysmosgraphic waveform. Each manufacturer uses unique proprietary algorithms to calculate the waveform displayed on the pulse oximeter. Standard pulse oximetry utilises two wavelengths of light in the red and infrared spectrum. Unlike the red wavelength, the absorbance of the infrared (IR) signal is relatively unaffected by changes in arterial oxygen saturation. Instead, the absorbance of the IR signal changes with the pulsations associated with blood volume in the peripheral vascular bed at the sensor site. With each heartbeat, the ventricle pumps blood into the periphery, increasing the pulse pressure in the arteries and arterioles and thus increasing the volume of blood under the sensor during systole. The reverse, decreases in pulse pressure and blood volume in the periphery, occurs
during dystole.

The photoplethysmosgraphic waveform displayed on some pulse oximeters represents the highly processed and filtered, pulsatile component of the IR signal over time, and, therefore, provides an indirect measure of blood volume or pulsatile strength under the sensor. Initially the photoplethysmosgraphic waveform was found to be useful as a simple indicator of the pulse oximeter signal integrity and changes in perfusion. Since then, clinicians have used the waveform display in a number of unique ways to obtain information on the physiological status of their patients. Clinicians observed, for example, that respiratory induced variations in the waveform amplitude (referred to as ∆POP) resemble the cyclical changes in pulse pressure and pulse volume that occur during the respiratory cycle, as measured with an arterial catheter (Figure 1). This is because the pumping action of the heart is directly influenced by relative changes in airway (intrapleural) pressure and blood pressure/blood volume.\cite{barach2000pulsus} 
%--
\begin{figure}[h!]
\centering
\includegraphics[scale=0.3]{figures/Seven.png}
\caption{Relation between respiratory variations in pulse oximetry plethysmographic waveform amplitude and arterial pulse pressure in ventilated patients. Adapted from Cannesson et al.,}
\label{fig:RIIVandPPG}
\end{figure}
%--
\section{Respiration Induced Intensity RIIV}
Based on the LBNP research, it is obvious that respiratory effort (Respiration Induved Intensity Veriation) is being observed during LBNP.
%--
\begin{figure}[h!]
\centering
\includegraphics[scale=.09]{Figures/MinuteOf_40mmPleth_OneMinute.png}
\caption{A minute of PPG signal from LBNP during -40 mm}
\label{fig:kian}
\end{figure}
%--
\begin{figure}[h!]
\centering
\includegraphics[scale=.09]{Figures/MinuteOf_40mmPleth_TwoCycles.png}
\caption{Two cardiac cycles from a minute of PPG signal from LBNP during -40 mm}
\label{fig:kian}
\end{figure}



\section{Pleth Veriablity Index}

PVI, a measurement  that automatically and continuously calculates the respiratory variations in the photoplethysmogram from data collected noninvasively via a pulse oximetry sensor. PVI visually correlates with ∆POP34 but uses a different algorithm to calculate the PVI value. PVI is a measure of the dynamic changes in the Perfusion Index (PI) that occur during one or more complete respiratory cycles. PI reflects the amplitude of the pulse oximeter waveform and is calculated as the pulsatile infrared signal (AC or variable component), indexed against the non-pulsitile infrared signal (DC or constant component). The infrared signal is used because it is less affected by changes in arterial saturation than the red signal.

\begin{figure}[h!]
\centering
\includegraphics[scale=.2]{Figures/PIVFundamental.png}
\caption{Graphic representation of raw infrared signal processed internally by the pulse oximeters, where AC represents the variable absorption of infrared light due to pulsating arterial inflow and DC represents the constant absorption of infrared light due to skin and other tissues.}
\label{fig:PIV}
\end{figure}

\begin{figure}[h!]
\centering
\includegraphics[scale=.2]{Figures/PeakDetectionFromThreeCyclesAt_40ForAMinute.png}
\caption{This Graph shows the peak detection algorithm applied on three cardiac cycles. As shown here, the algorithm was capable of detection systolic and diastolic peaks.}
\label{fig:PIV}
\end{figure}

\begin{figure}[h!]
\centering
\includegraphics[scale=.2]{Figures/AMinutePPGFingerRIIV_40_RIIV.png}
\caption{This Graph shows the peak detection algorithm applied on a minute of PPG signal. Peaks were detected from finger ppg sensor. The peaks in this graph corresponds to RIIV }
\label{fig:PIV}
\end{figure}




\begin{figure}[h!]
\centering
\includegraphics[scale=.2]{Figures/PPGFlexAMinuteAt_40RIIV.png}
\caption{This Graph shows 12 respiratory cycles. This is a PPG Flex signal that was obtained from the a minutes of data at -40mm. A low pass filter was applied and then the RIIV was extracted. Part of the PPG signal was reduced as the blood perfusion changed in the peripheral tissue, and thus the amplitude of the signal changes. The aptitude of the PPG signal ini figure 6, is a filtered and amplified PPG signal, thus the amplitude of this signal does no correspond to the intra arterial pressure.  }
\label{fig:PIV}
\end{figure}

\clearpage

\begin{figure}[h!]
\centering
\includegraphics[scale=.2]{Figures/AMinutePPGFingerPeaks_40_ThreePeaks_cardiacCycle.png}
\caption{This Graph shows the peak detection algorithm applied on three cardiac cycles. Peaks were detected from finger ppg sensor. detects systolic and diastolic peaks}
\label{fig:PIV}
\end{figure}




\begin{figure}[h!]
\centering
\includegraphics[scale=.2]{Figures/AMinutePPGFingerPeaks_40.png}
\caption{This Graph shows the peak detection algorithm applied on three cardiac cycles. The graphs shows the performance of the algorithm for a minute of data. Peaks were detected from finger ppg sensor. and it successflly detects systolic and diastolic peaks}
\label{fig:PIV}
\end{figure}



\vspace*{-2mm}
\bibliography{bib/med,bib/nn}

\begin{correspondence}
Hans A. Kestler\\
Dept. of Medicine II / University Hospital Ulm\\
Robert-Koch-Str. 8 / D-89081 Ulm / Germany\\
tel./fax: ++49-731-502-4437/4442\\
h.kestler@ieee.org
\end{correspondence}

\end{document}

