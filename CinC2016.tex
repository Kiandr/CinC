% This file has been slightly modified from Hans Kestler's original to reflect
% changes in the CinC style file (cinc.cls), in which the width of the text
% columns has been slightly reduced.  Accordingly, the size of figure 2 has
% been reduced so that the entire paper will fit into four pages as in the
% original. --GBM
\documentclass[twocolumn]{cinc}


\usepackage{epsfig}



\def\IR{{\rm I\! R}}
\def\IN{{\rm I\! N}}
\def\IF{{\rm I\! F}}
\def\IP{{\rm I\! P}}
\def\IK{{\rm I\! K}}
\def\IZ{{\rm Z\!\! Z}}
\def\IQ{\,{\rm Q\!\!\! \vrule width 0.5pt height 7pt depth 0pt }\;\,}
\def\IC{\,{\rm C\!\!\!\>\! \vrule width 0.5pt height 7pt depth 0pt }\;\,}
\def\argmin{\rm argmin}

\sloppy



\begin{document}

\bibliographystyle{cinc}

\title{Photoplethysmography  as Primary Tool for Progressive Haemorrhage Assesment During Progressive Lower Body Negative Pressure}
\author{Kian~Davoudi$^{1}$, Kouhyar~Tavakolian$^2$, Bozena~Kaminska$^1$ 
\\[1em]
        $^1$Simon Fraser University, Vancouver, Canada \\[.25ex]
        $^2$University of North Dakota, North Dakota, United Stated of Amercia\\[1em]
}

\maketitle






\begin{abstract}
Lack of a precise index to identify degree of  haemorrhage prior cardiovascular collapse is being considered a driving cause of fatality in patients experiencing progressive haemorrhage. The traditional clinical haemorrhage assessment consists of conventional beat-to-beat heart rate and arterial blood pressure measurement. But, as shown in xx, prior cardiovascular collapses,  these indexes continue projecting no significant shift in their status. Hence, Convertino and Tavakolian have shown that SCG and PPG indeed provide such a window to alarm cardiovascular collapse prior radical projection on conventional measurements. Hence, As I shown in xxx, a mechanically flexible PPG sensor was designed and optimized for continuous monitoring of cardiovascular and respiratory performance. But, the reliability of the this sensor for calcification of stages of haemorrhage, was not assessed. Thus, I aimed to resolve this issue,by conducting Lower Body Negative Pressure Test to stimulate haemorrhage in 10 awaked patients and compared the reliability of PPG signal captured from mechanically flexible sensor, where due to arterial blood loss in upper body was diminished, with conventional PPG captured from FDA approved pulse oximeter. Results from Bland and Altman analysis of  a minutes of PPG signal at -50 mm for 10 subjects were, 


\end{abstract}


\section{Introduction}

High-resolution electrocardiography is used for the detection of fractionated micropotentials, which serve
as a noninvasive marker for an arrhythmogenic substrate and for an increased risk for malignant 
ventricular tachy\-arrhythmias.
Ventricular late potential analysis (VLP) is herein the generally accepted noninvasive method to identify patients
with an increased risk for reentrant ventricular tachycardias and for risk stratification after myocardial
infarction \cite{med:gomes89,med:hoher91,med:simson81}. 
Techniques commonly applied in this purely time--domain based analysis are
signal-averaging, high-pass filtering and late potential analysis of the terminal part of the QRS complex.
The assessment of VLP's depends on three empirically defined limits of the total duration of the QRS and the 
duration and amplitude of the terminal low--amplitude portion of the QRS complex \cite{med:breithardt86,med:breithardt91}.

In this study we investigated the neural realization of the time domain based analysis.
As the process of decision making is separated into two phases: (a) feature extraction and (b) categorization
these two phases were modeled with two different multilayer perceptron neural networks. 
This results in four different combinations of the standard and neural approaches. These are compared.



\section{Notation}

A formalized description of the three features used for time domain analysis of signal averaged
ECGs is as follows:

\vspace*{-2cm}
\centerline{\epsfig{file=figures/qrs1, width=10cm}}
\vspace*{-7.5cm}        
{\small
\begin{itemize}
\item QRSd (QRS duration):
\[
 QRSD := QRS_{offset} - QRS_{onset}
\]
\item RMS (Time A: $A:= QRS_{offset} - 40ms$): 
\[
RMS := \sqrt{\frac{1}{QRS_{offset} - A} \sum_{i=A}^{QRS_{offset}} V_i^2}
\]
\item LAS (Duration of the low amplitude signal below 40$\mu$V):
\[
LAS := QRS_{offset} - argmax \{ i \; | \; V_i \ge 40 \mu V \}
\] 
\end{itemize}}
    

\section{Subject data}

We compared a group of 51 healthy subjects (group A) with 44 cardiac patients 
at a high risk for malignant ventricular arrhythmias (group B, VT patients). 
All healthy 
volunteers (mean age 24.0$\pm$4.1 years) had a normal resting ECG and a normal 
echocardiogram, and no cardiac symptoms or coronary risk factors. 
The patients with a
high--risk for malignant ventricular arrhythmias (mean age 61.2$\pm$8.9 years) were selected 
from our electrophysiologic database. Inclusion criteria were the presence of coronary 
artery disease, a previous myocardial infarction, a history of at least one symptomatic 
arrhythmia, and inducible sustained ventricular tachycardia ($>$ 30 seconds) at electrophysiologic 
testing. Patients with bundle branch block or atrial fibrillation were excluded.
All patients of group B underwent coronary angiography and programmed right ventricular 
stimulation due to clinical indications. Stimulation was done from the right apex and the 
right outflow tract. The stimulation protocol included up to 3 extrastimuli during sinus 
rhythm and at baseline pacing with a cycle length of 500 ms, and a maximum of 2 extrastimuli 
at baseline pacing with cycle lengths of 430 ms, 370 ms, and 330 ms.
Group B consisted of 10 patients with single vessel disease, 17 patients with double vessel 
disease, and 17 patients with triple vessel coronary artery disease. 
Nineteen patients had a previous posterior 
infarction, 14 patients had a previous anterior infarction, and 11 patients had both a previous 
anterior and a previous posterior infarction. Mean left ventricular ejection fraction 
was 44.0\%$\pm$14.9\%. Forty-one patients had a documented episode of spontaneous, sustained 
ventricular tachycardia or ventricular fibrillation. Out of the remaining 
three patients, 1 patient had syncopes 
and non--sustained ventricular tachycardias on Holter monitoring, and 2 patients had syncopes 
of presumed cardiac origin.
 


\section{Methods}


\subsection{ECG recordings}
High--resolution signal averaged electrocardiograms  
were recorded during sinus rhythm from bipolar orthogonal $X$, $Y$, $Z$ leads. 
Before ECG recording antiarrhythmic drugs were stopped for at least four half--lives.
The skin was carefully prepared and recordings were done with the subjects in reclining 
position in a Faraday cage. Sampling rate was 2000 Hz, A/D resolution was 16 bit, and an 
analog bandpass filter of 0.05-300 Hz was used (anti--aliasing).  The ECG's were recorded 
with the Predictor system (Corasonix Inc., Oklahoma, USA).   
The three leads were combined into a vectormagnitude signal $V=\sqrt{X^2 + Y^2 + Z^2}$ and
bidirectionally filtered with a 4--pole Butterworth filter (40--250 Hz). 



\subsection{Overview}

An overview of the evaluation process is given in Fig. \ref{overview}. From the signal averaged ECG
estimated and visually corrected QRS--onset and --offset points are used as target values for the 
training of the feature extraction multilayer perceptron. This neural net is trained to predict these
points from the bidirectional filtered ECG. From these points the features QRSd, LAS and RMS are
calculated. These serve as the input to the next stage. The standard approach of obtaining these
values consists of inspecting and if necessary correcting an estimation given by a conventional
signal processing algorithm, in this case the Predictor$^\copyright$ system, see \cite{med:breithardt91}. 
The difficulty with this approach lies in finding the QRS--offset. 

The next stage consists of categorization of the ECGs based upon the three extracted features. This
is done by another multilayer perceptron network trained to predict the correct group status, see Section 2.
The same is done with empirically found and defined limits of QRSd, LAS and RMS \cite{med:breithardt91}.
More formally, the standard method classifies into VLP positive if 2 of 3 
criteria are met: QRSd $>$ 114ms, RMS $<$ 20$\mu$V, LAS $>$ 38ms. Evaluation of this categorization
is against group status.

\begin{figure*}[h] 
\vspace*{-2cm}
\centerline{\epsfig{file=figures/wholesys2a, width=17cm, clip=}}
\caption{Overview of the decision process incorporating neural and standard methods.}
\label{overview} %\vspace*{3mm} 
\end{figure*}



\subsection{Multilayer perceptron}

It has been proved that multilayer perceptron (MLP) networks built of 3 layers
of artificial neurons are capable of approximating any continuous function.
Together with the backpropagation algorithm for weight adaptation MLP's
are powerful schemes for function approximation and categorization \cite{nn:haykin99}.
\begin{figure}[ht]
\vspace {-.5cm} 
\centerline{\epsfig{file=figures/nn2c, width=6cm}}
\vspace*{.5cm}
\centerline{\hspace*{0cm}\epsfig{file=figures/nn2d, width=6cm}}
\vspace{.3cm}
\caption{Multilayer perceptron net and backpropagation.}
\label{mlp1} \vspace*{-15mm} 
\end{figure}


\clearpage



\subsection{Training of QRS--onset and --offset}

The training of the feature extracting MLP was done by sliding an empirically
determined window of 16ms, i.e. 32 inputs, over the bidirectionally filtered QRS complex.
The target value was either a "1" if the QRS--onset or --offset was inside the window
or a  "0" if it was outside. This results in a balanced training set: ratio of "1" to "0" target values is 2.8.
The onsets and offsets were visually determined.
The best performing net was of size 32--8--1. All neurons
had a Fermi transfer function.
 
\begin{figure}[ht] 
\centerline{\hspace*{0cm}\epsfig{file=figures/winmove, width=7cm, clip=}}
\vspace*{-3cm} 
\caption{Training of the QRS--onset and --offset. MLP input: 32 sample points within the
input window. MLP target: value of the characteristic function $y(t)$.}
\label{mlp2} \vspace*{3mm} 
\end{figure}
       
        
\begin{figure}[ht] 
\centerline{\hspace*{-5mm}\epsfig{file=figures/snnsout1b, width=5.3cm, height=2cm, clip=}}
\caption{Sample net output for the QRS--onset, QRS--offset training.}     
\label{mlp3} \vspace*{3mm} 
\end{figure}



\subsection{MLP training with calculated features}

For each QRS complex the three features QRSd, RMS and LAS (see Section 2) were calculated.
QRS--onset and --offset were either estimated by an expert or by the feature extraction MLP
network (previous subsection). MLP networks (3 input neurons, 8 hidden neurons and 1 output
neuron) were trained to predict the patient's group status.




\vspace*{-2mm}
\section{Results and conclusion}

The classification process of signal averaged ECGs is separated into two phases: (a) feature extraction
and (b) categorization according to the found feature values. These steps were realized both with the
neural network (MLP) and the standard approach (STD). Four combinations of the methods are possible.
Combinations involving a neural network were evaluated applying a 5--fold crossvalidation:
    %\vspace*{1cm}
\begin{center}
\small
\begin{tabular}{|c||c|c|c|c|c|}\hline
           & Acc  & Sens & Spec & PPV   & NPV   \\ \hline \hline
STD \& STD & 0.789     & 0.409       & 0.960       & 0.928 & 0.731 \\ \hline
MLP \& STD & 0.754     & 0.513       & 0.906       & 0.918 & 0.696 \\ \hline
STD \& MLP & 0.851     & 0.736       & 0.949       & 0.925 & 0.806 \\ \hline 
{\bf MLP \& MLP} & {\bf 0.846}     &{\bf 0.745}       & {\bf 0.933}       & {\bf 0.906} & {\bf 0.809}\\ \hline
\end{tabular}
\end{center}
{ This feasibility study gives very positive initial results on using a neural
network for the classification of signal averaged ECGs. Furthermore the feature extracting MLP
makes the system independent of a correction of QRS--onset and --offset.}


\vspace*{-2mm}
\bibliography{bib/med,bib/nn}

\begin{correspondence}
Hans A. Kestler\\
Dept. of Medicine II / University Hospital Ulm\\
Robert-Koch-Str. 8 / D-89081 Ulm / Germany\\
tel./fax: ++49-731-502-4437/4442\\
h.kestler@ieee.org
\end{correspondence}

\end{document}

